% Title page
\makeatletter
\begin{titlepage}
	\begin{center}
		\vspace*{4cm}

		\huge
		\@title

		\vspace*{2cm}

		\LARGE
		\@author

		\vspace*{1cm}
		\@date

		\vfill

		\vspace*{1cm}
		\normalsize
		An Undergraduate Thesis\\
		Honors College and College of Engineering\\
		Mentored by Dr. Mike Rosulek
		\vspace*{2cm}
	\end{center}
\end{titlepage}
\makeatother

% Second page
\begin{abstract}
	Multi-party computation (MPC) is a field of study focused on devising cryptographic protocols that allow participants to learn the output of some function of their private inputs without trusting a third party to perform the computation. This is usually done at a large scale between data centers, with little emphasis on individuals' devices or mobile hardware. In this paper, I present a proof-of-concept implementation of two-party computation on a commodity smartphone paired with a low-power field-programmable gate array (FPGA). I compare the performance and power consumption of the system between a software-only setup and a setup with the FPGA coprocessor used for acceleration. I find a calculated $62\times$ speed improvement assuming a saturated serial connection, and no significant difference in the smartphone's battery life.
\end{abstract}

\vfill
\tableofcontents
