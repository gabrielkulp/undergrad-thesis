% Title page
\makeatletter
\begin{titlepage}
	\begin{center}
		\vspace*{4cm}

		\huge
		\@title

		\vspace*{2cm}

		\large
		\@author

		\vfill
		
		An Undergraduate Thesis
		\vspace*{2cm}
	\end{center}
\end{titlepage}
\makeatother



% Second page
\tableofcontents

\vfill

\begin{abstract}
Multi-party computation (MPC) is a cryptographic protocol that allows parties to learn the output of some function of their private inputs without trusting a third party to perform the computation. This is usually done at a large scale between data centers, with little emphasis on individuals' devices or mobile hardware. In this paper, I present a proof-of-concept implementation of two-party computation on a commodity smartphone paired with a low-power FPGA\@. I compare the performance and power consumption of the system with a software-only setup and with the FPGA connected for acceleration.
\end{abstract}
