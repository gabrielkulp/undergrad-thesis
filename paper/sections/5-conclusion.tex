\section{Conclusion}
\subsection{State of the Field}
Most focus is on data centers and there doesn't seem to be a lot of work on low-power devices like smartphones and \textit{small} FPGAs. I hope that doing anything with these devices makes me a more valuable member of the community. Developing in constrained environments inevitably helps the unconstrained ones, too.

\subsection{Viability}
There's work on cloud computing \comment{like Mike's work on neural nets}, so I want to show that that could be applied to our mobile devices. Obviously we wouldn't really have an FPGA dangling from our phones, but dedicated hardware could be cool. \comment{Isn't that just AES-NI, though? I need to draw a distinction.}

I think this could be viable with a faster phone-to-FPGA connection. We might get to a situation where throughput is the bottleneck, and then with 5G or whatever then that's probably a good place to be.

Attaching a low-end FPGA has significant performance potential for the MPC workload on a low-end mobile processor.

In conclusion, MPC is a kind of cryptography that allows a whole new class of problems to be solved, but it takes a lot of computing power. MPC could be adopted more readily if there was an easy way to offload that work onto specially-designed hardware. In my project, I conclude that this coprocessor is a feasible architecture to this end with significant performance potential.
