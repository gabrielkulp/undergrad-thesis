\section{Conclusion}
My vision is to make privacy-preserving computations more common. In the United States, HIPAA and FERPA laws protect healthcare and student records respectively, but these laws prevent analysis that could have great value to society. MPC allows calculations on these data that no single party could ethically or legally compute on their own.

Most academic focus in the field of MPC is on data centers, with many fewer published attempts at enabling low-power devices like smartphones and small FPGAs to participate. Working with these low-power devices has value to the cryptography community: developing in constrained environments inevitably contributes to the less-constrained ones, and the wide-spread nature of mobile devices offers many opportunities for real-world applications of privacy-preserving cryptography.

Our devices already have efficient hardware support for some cryptographic operations, opening the doors to online banking, secure video calling, and other private transactions. I would like to see mobile devices that can efficiently do MPC with each other or cloud services to open the doors to even more private applications. 

This thesis presents a feasible architecture with significant performance potential for low-end mobile devices. With a faster CPU-to-coprocessor connection or the coprocessor instructions implemented as en extension of existing mobile SoC instructions, mobile MPC could be constrained only by the speed of the network connection. Future work promises to be exciting.
