\section{Introduction}

\subsection{Motivation}
Multi-party computation (MPC) solves problems that are otherwise impossible without trust. The classic example\cite{YaoGC} is that two millionaires at a party want to find out who has more money, but they don't want to divulge how much money they have. They could whisper in the butler's ear, but he might tell someone later or give the wrong answer, so they'd rather not trust anybody. MPC allows them to answer their question without trusting each other or the butler.

A real example from Denmark in 2008 involved the process of finding a fair price for sugar beets after EU market changes: for each potential price, the buyer specifies how much they are willing to buy and the seller specifies how much they are willing to sell. The ``market clearing price'' is then derived from a computation on these data. While this could have been done via a trusted party, the single Danish sugar beet buyer and the farmer's union were not suitable choices, and hiring a consultant would have been too expensive. The solution was to perform a multi-party computation such that the optimal price could be determined without the buyer or sellers revealing private information.\cite{beets}

In the modern cloud-computing model, providers compute on client data using proprietary algorithms. When clients send data to be processed, it is also available to the service provider for logging and analysis. This violation of privacy serves as a building block of the world of commercial IoT, cloud services, and centralized machine learning. MPC provides a cryptographic solution to this problem, removing the need to share data or trust a third party, while still computing the same results.

Support for efficient MPC in mobile devices could open the doors to many other privacy- and security-focused improvements to how our computers communicate. For example, a user could evaluate a pre-trained machine learning model without revealing their data to the service provider and without the service provider revealing their model\cite{NeuralNets}. This would, for example, allow a user to take a photograph and have it classified by a private algorithm without providing the image in the clear to the owner of the algorithm.

One obstacle to widespread adoption of MPC is the poor efficiency of execution. The protocol's cryptographic overhead slows the computation by several orders of magnitude (see Sec.~\ref{sec:gc}). To address this, I propose a coprocessor to accelerate the client's workload. This coprocessor is attached to a smartphone to handle the most power-hungry aspects of the calculation more efficiently than the phone's built-in processor. In theory, this technique could be used to make MPC a common task in the same way that other cryptographic operations (like the TLS handshake) are widespread and efficient.

\subsection{Scope}
In this thesis, I implement Yao's Garbled Circuits in software, and compare its system resource utilization (CPU time and RAM usage) and power consumption to a second, interoperable implementation in which the smartphone interfaces with an FPGA\@. It is out of scope to draw comparisons to other implementations or design a practical user interface.

I will not perform hardware modifications to the smartphone, nor design a new device as a stand-in; this limits the throughput of the data link between the smartphone and FPGA.

I chose to forgo some optimizations because they would have taken too long to learn and implement. In particular, half-gates and garbled gadgets are out of scope for this project, but are prime candidates for future work.

The co-processor architecture on the FPGA is tailored to the small and low-power chip selected for its open-source\cite{IceStorm} compatibility. More details on these design decisions are in Sec.~\ref{sec:hw}.
